\begin{center}
  \large\textbf{ABSTRAK}
\end{center}

\addcontentsline{toc}{chapter}{ABSTRAK}

\vspace{2ex}

\begingroup
% Menghilangkan padding
\setlength{\tabcolsep}{0pt}

\noindent
\begin{tabularx}{\textwidth}{l >{\centering}m{2em} X}
  Nama Mahasiswa    & : & \name{}         \\

  Judul Tugas Akhir & : & \tatitle{}      \\

  Pembimbing        & : & 1. \advisor{}   \\
                    &   & 2. \coadvisor{} \\
\end{tabularx}
\endgroup

% Ubah paragraf berikut dengan abstrak dari tugas akhir
Pengembangan kursi roda otonom telah menjadi semakin penting dalam memberikan mobilitas dan kemandirian yang ditingkatkan bagi individu dengan mobilitas terbatas. Studi ini mengusulkan pengembangan sistem kursi roda otonom berbasis YOLOv8 untuk menghindari obstacle, khususnya fokus pada deteksi obstacle manusia. Dengan memanfaatkan kemampuan deteksi objek yang canggih dari YOLOv8, sistem yang diusulkan bertujuan untuk mendeteksi dan menghindari obstacle manusia secara efektif. Sistem tersebut mendeteksi manusia melalui video menggunakan Single Based Computer Intel NUC dan Kamera. Obstacle yang dideteksiakan membuat SBC mengirim perintah ke ESP32 untuk menjalankan motor untuk melakukan manuver penghindaran. Evaluasi sistem meliputi pengujian komprehensif di berbagai lingkungan dan konteks navigasi untuk memvalidasi efektivitas dan kehandalan dalam aplikasi dunia nyata. Penelitian ini berkontribusi pada kemajuan teknologi kursi roda otonom, dan keselamatan bagi individu dengan keterbatasan mobilitas

% Ubah kata-kata berikut dengan kata kunci dari tugas akhir
Kata Kunci: Kursi Roda Otonom, \emph{YOLOv8}, Intel NUC, ESP32, Deteksi Manusia, Bantuan Mobilitas
