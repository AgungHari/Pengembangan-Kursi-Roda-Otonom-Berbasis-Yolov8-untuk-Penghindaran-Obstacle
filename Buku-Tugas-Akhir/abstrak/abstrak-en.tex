\begin{center}
  \large\textbf{ABSTRACT}
\end{center}

\addcontentsline{toc}{chapter}{ABSTRACT}

\vspace{2ex}

\begingroup
% Menghilangkan padding
\setlength{\tabcolsep}{0pt}

\noindent
\begin{tabularx}{\textwidth}{l >{\centering}m{3em} X}
  \emph{Name}     & : & \name{}         \\

  \emph{Title}    & : & \engtatitle{}   \\

  \emph{Advisors} & : & 1. \advisor{}   \\
                  &   & 2. \coadvisor{} \\
\end{tabularx}
\endgroup

% Ubah paragraf berikut dengan abstrak dari tugas akhir dalam Bahasa Inggris
\emph{The development of autonomous wheelchairs has become increasingly vital in providing enhanced mobility and independence for individuals with limited mobility. This study proposes the development of an autonomous wheelchair system based on YOLOv8 for obstacle avoidance, specifically focusing on human obstacle detection. Utilizing the advanced object detection capabilities of YOLOv8, the proposed system aims to effectively detect and avoid human obstacles. The system detects humans through video using an Intel NUC Single Board Computer and camera. Detected obstacles trigger the SBC to send commands to the ESP32 to drive motors for evasion maneuvers. System evaluation includes comprehensive testing in various environments and navigation contexts to validate effectiveness and reliability in real-world applications. This research contributes to the advancement of autonomous wheelchair technology and safety for individuals with mobility limitations}

% Ubah kata-kata berikut dengan kata kunci dari tugas akhir dalam Bahasa Inggris
Autonomous Wheelchair, YOLOv8, Intel NUC, ESP32, Human Detection,
 Mobility Aid.
