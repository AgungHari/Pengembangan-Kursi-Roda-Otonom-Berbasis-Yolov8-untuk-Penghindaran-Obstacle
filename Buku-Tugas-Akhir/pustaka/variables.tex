% Atur variabel berikut sesuai namanya

% nama
\newcommand{\name}{I Gst Ngr Agung Hari Vijaya Kusuma}
\newcommand{\authorname}{Agung Hari, I Gst Ngr}
\newcommand{\nickname}{Agung}
\newcommand{\advisor}{Dr. Eko Mulyanto Yuniarno, S.T.,M.T.}
\newcommand{\coadvisor}{Wernher von Braun, S.T., M.T}
\newcommand{\examinerone}{Dr. Galileo Galilei, S.T., M.Sc}
\newcommand{\examinertwo}{Friedrich Nietzsche, S.T., M.Sc}
\newcommand{\examinerthree}{Alan Turing, ST., MT}
\newcommand{\headofdepartment}{Prof. Albus Percival Wulfric Brian Dumbledore, S.T., M.T}

% identitas
\newcommand{\nrp}{07211940000073}
\newcommand{\advisornip}{19680601199512 1 009}
\newcommand{\coadvisornip}{}
\newcommand{\examineronenip}{}
\newcommand{\examinertwonip}{}
\newcommand{\examinerthreenip}{}
\newcommand{\headofdepartmentnip}{}

% judul
\newcommand{\tatitle}{PENGEMBANGAN KURSI RODA OTONOM BERBASIS  \emph{YOLOV8} UNTUK PENGHINDARAN \emph{OBSTACLE}}
\newcommand{\engtatitle}{\emph{Development of YOLOv8-based Autonomous Wheelchair for Obstacle Avoidance.}}

% tempat
\newcommand{\place}{Surabaya}

% jurusan
\newcommand{\studyprogram}{Teknik Komputer}
\newcommand{\engstudyprogram}{Computer Engineering}

% fakultas
\newcommand{\faculty}{Teknologi Elektro dan Informatika Cerdas}
\newcommand{\engfaculty}{Intelligence Electrics and Informatics Technology}

% singkatan fakultas
\newcommand{\facultyshort}{FTEIC}
\newcommand{\engfacultyshort}{ELECTICS}

% departemen
\newcommand{\department}{Teknik Komputer}
\newcommand{\engdepartment}{Computer Engineering}

% kode mata kuliah
\newcommand{\coursecode}{EC234801}
