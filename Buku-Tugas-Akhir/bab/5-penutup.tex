\chapter{PENUTUP}
\label{chap:penutup}

% Ubah bagian-bagian berikut dengan isi dari penutup

\section{Kesimpulan}
\label{sec:kesimpulan}

Berdasarkan hasil pengujian yang telah dilakukan, dapat ditarik kesimpulan sebagai berikut:

\begin{enumerate}[nolistsep]
  \item Model dengan Metrics tertinggi yang telah di-training dengan berbagai konfigurasi dan arsitektur adalah model dengan skor mAP di IoU 0.5 tertinggi sebesar 94.10\% yang menggunakan arsitektur SSD-Mobilenet-V2 FPN-Lite 320, skor tersebut lebih tinggi dibandingkan penelitian sebelumnya yang digunakan sebagai referensi
  \item Performa NUC dalam pengujian FPS menghasilkan Nilai yang lebih rendah ketimbang Laptop pribadi Penulis dengan selisih 7.029
  \item Rata - rata delay yang didapatkan pada pengujian adalah sekitar 0.248 dan rata- rata nilai inference yang didapatkan
  \item Hasil Performa Deteksi menunjukan hasil yang memuaskan dalam 10 sampel pengujian. Dengan presentasi keberhasilan sebesar 100\%

\end{enumerate}

\section{Saran}
\label{chap:saran}

Untuk pengembangan lebih lanjut pada penelitian selanjutnya, adapun saran yang bisa diberikan antara lain:

\begin{enumerate}[nolistsep]

  \item Variasi dataset yang lebih ditingkat untuk meningkatkan performa deteksi

  \item Menggunakan SBC yang memiliki performa yang lebih baik untuk fps yang lebih tinggi

  \item Meningkatkan performa grid dengan mengubah ukuran 10x10 menjadi 100x100 atau lebih untuk hasil lebih detail

\end{enumerate}
